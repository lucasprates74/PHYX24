\documentclass[11pt]{article}
%============= Imports =============
\usepackage{physics}
\usepackage{amsmath,amsthm,amssymb}
\usepackage[dvips,letterpaper,margin=1.1in,bottom=0.7in]{geometry}
\usepackage{hyperref}
\usepackage{enumitem}
\usepackage{charter}
\usepackage{tabstackengine}
\usepackage{fancyhdr}
\usepackage{graphicx}
\usepackage{subfig}
\usepackage{siunitx}
\usepackage{float}
\usepackage{listings}
\usepackage{xcolor}
\usepackage{natbib}
\bibliographystyle{apa-good}
%============= Page style =============
\pagestyle{fancy}
\newcommand{\CourseDef}{PHY424} %Course
\newcommand{\TitleDef}{Statistics and dynamics of vertically vibrated granular chains} %Title
\newcommand{\NumberDef}{1006154625}
\newcommand{\AuthorDef}{Lucas Prates} %Author (Name)
\rhead{\AuthorDef}
\chead{\TitleDef}
\lhead{\CourseDef}
\definecolor{codegreen}{rgb}{0,0.6,0}
\definecolor{codegray}{rgb}{0.5,0.5,0.5}
\definecolor{codepurple}{rgb}{0.58,0,0.82}
\definecolor{backcolour}{rgb}{0.95,0.95,0.92}
\lstdefinestyle{mystyle}{
    backgroundcolor=\color{backcolour},   
    commentstyle=\color{codegreen},
    keywordstyle=\color{magenta},
    numberstyle=\tiny\color{codegray},
    stringstyle=\color{codepurple},
    basicstyle=\ttfamily\footnotesize,
    breakatwhitespace=false,         
    breaklines=true,                 
    captionpos=b,                    
    keepspaces=true,                 
    numbers=left,                    
    numbersep=5pt,                  
    showspaces=false,                
    showstringspaces=false,
    showtabs=false,                  
    tabsize=2
}

\lstset{style=mystyle}
%============= Matrix Commands =============
\setstackEOL{;}% row separator
\setstackTAB{,}% column separator
\setstacktabbedgap{2ex}% inter-column gap
\setstackgap{L}{1.0\normalbaselineskip}% inter-row baselineskip
\let\bmatrix\bracketMatrixstack
\let\pmatrix\parenMatrixstack
\let\dmatrix\vertMatrixstack
\newcommand{\declarecommand}[1]{\providecommand{#1}{}\renewcommand{#1}}
%============= Other Commands =============
\declarecommand{\R}{\mathbb{R}}
\declarecommand{\Q}{\mathbb{Q}}
\declarecommand{\Z}{\mathbb{Z}}
\declarecommand{\N}{\mathbb{N}}
\declarecommand{\C}{\mathbb{C}}
\declarecommand{\emptyset}{\varnothing}
\newcommand{\mat}[1]{\begin{bmatrix}#1\end{bmatrix}}
%%[label=(alph*)]
%============= Title Config =============
%\title{\CourseDef\space\TitleDef}
%\author{\AuthorDef \\ \NumberDef}
\begin{document}
\thispagestyle{plain}
{\noindent\Huge\bf  \\[0.5\baselineskip] {\fontfamily{cmr}\selectfont  \TitleDef}         }\\[2\baselineskip] % Title
{ {\bf \fontfamily{cmr}\selectfont \CourseDef}\\ {\textit{\fontfamily{cmr}\selectfont     \today}}}\hspace{150pt}    {\large \textsc{\AuthorDef}} % Author name
\\[1.4\baselineskip]
\vspace{-15pt}

%\begin{figure}[h]%
%    \centering
%    \subfloat{{\includegraphics[scale=0.56]{Kp_Upstream_figure.png} }}%
%    \qquad
%    \subfloat{{\includegraphics[scale=0.56]{Kp_Downstream_figure.png} }}%
%    \caption{Time-averaged $K_{\rho}^U$ (a) and $K_{\rho}^D$ (b) for each simulation. The time average was taken        on the last $54t_0$ ($\sim20$ minutes of simulation time) of the simulation.}%
%    \label{fig:diffus}%
%\end{figure}
%============= Start File =============

\begin{abstract}
    This experiment ventured to study the statistics and dynamics of vertically vibrated
    granular chains and consisted of two phases. The first phase studied the opening times
    of a trefoil knot initially located in the center of the chain. The second phase studied the 
    equilibrium radius of gyration of the chain after the dissipation of transient motion.\\

    In the first phase, I collected 30-60 opening times for each chain length $40\leq N \leq 140$, 
    with intervals of $20$ beads. The data results are in tables 1-12 (pg 7-20), but note 
    that tables 1, 2, and 5 have been discarded due to an inconsistent shaker plate amplitude 
    (pg 21). After ommiting the long tail end times, I found that the mean opening times for 
    each chain length $N$ were given by $\tau_{\text{avg}} \propto (N-N_0)^{\delta}$ with 
    $\delta = 2.01 \pm 0.07$ with $\chi_{\text{red}}^2=1.04$ and CDF = 38.4\% (pg 25-28). I also 
    plotted the empirical survival probability for each chain length data set (pg 23) and verified 
    that they come from the same distribution using the Komogorov Smirnov test (pg 44).\\

    For the second phase, I searched for a relationship between equilibrium radius of gyration 
    and $N$. I recorded the motion of the vibrating ball chains for $40\leq N \leq 140$ with a 
    20 bead interval and also for $N=50, 55$ (pg 32-34 and 37-38) and computed the radius of 
    gyration at each 2 second interval. For each trial, I plotted 
    the radius of gyration for the last 60 seconds of motion, and plotted the distributions of 
    these radii of gyration, but trials 2, 4, and 8 were discarded (pg 40-41). At equilibrium, the chains tended to form spirals. I found 
    that there were 3 regimes of equilibrium motion (pg 42-43)
    
    \begin{itemize}
        \item $N<60$: spirals were unable to form
        \item $60<N<80$: loose spiral formation
        \item $80<N<140$: area filling spiral formation
    \end{itemize}

    Due to the complex nature of the equilibrium radius of gyration vs $N$ relationship, more data 
    is required in order find a best fit curve for each regime.
\end{abstract}
\end{document}
