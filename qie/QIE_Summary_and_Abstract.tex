\documentclass[11pt]{article}
%============= Imports =============
\usepackage{physics}
\usepackage{amsmath,amsthm,amssymb}
\usepackage[dvips,letterpaper,margin=1.1in,bottom=0.7in]{geometry}
\usepackage{hyperref}
\usepackage{enumitem}
\usepackage{charter}
\usepackage{tabstackengine}
\usepackage{fancyhdr}
\usepackage{graphicx}
\usepackage{subfig}
\usepackage{siunitx}
\usepackage{float}
\usepackage{listings}
\usepackage{xcolor}
%============= Page style =============
\pagestyle{fancy}
\newcommand{\CourseDef}{PHY424} %Course
\newcommand{\TitleDef}{Violation of Bell's inequality in the polarizations of down converted photon pairs} %Title
\newcommand{\NumberDef}{1006154625}
\newcommand{\AuthorDef}{Lucas Prates} %Author (Name)
\rhead{\AuthorDef}
\chead{\TitleDef}
\lhead{\CourseDef}
\definecolor{codegreen}{rgb}{0,0.6,0}
\definecolor{codegray}{rgb}{0.5,0.5,0.5}
\definecolor{codepurple}{rgb}{0.58,0,0.82}
\definecolor{backcolour}{rgb}{0.95,0.95,0.92}
\lstdefinestyle{mystyle}{
    backgroundcolor=\color{backcolour},   
    commentstyle=\color{codegreen},
    keywordstyle=\color{magenta},
    numberstyle=\tiny\color{codegray},
    stringstyle=\color{codepurple},
    basicstyle=\ttfamily\footnotesize,
    breakatwhitespace=false,         
    breaklines=true,                 
    captionpos=b,                    
    keepspaces=true,                 
    numbers=left,                    
    numbersep=5pt,                  
    showspaces=false,                
    showstringspaces=false,
    showtabs=false,                  
    tabsize=2
}

\lstset{style=mystyle}
%============= Matrix Commands =============
\setstackEOL{;}% row separator
\setstackTAB{,}% column separator
\setstacktabbedgap{2ex}% inter-column gap
\setstackgap{L}{1.0\normalbaselineskip}% inter-row baselineskip
\let\bmatrix\bracketMatrixstack
\let\pmatrix\parenMatrixstack
\let\dmatrix\vertMatrixstack
\newcommand{\declarecommand}[1]{\providecommand{#1}{}\renewcommand{#1}}
%============= Other Commands =============
\declarecommand{\R}{\mathbb{R}}
\declarecommand{\Q}{\mathbb{Q}}
\declarecommand{\Z}{\mathbb{Z}}
\declarecommand{\N}{\mathbb{N}}
\declarecommand{\C}{\mathbb{C}}
\declarecommand{\emptyset}{\varnothing}
\newcommand{\mat}[1]{\begin{bmatrix}#1\end{bmatrix}}
%%[label=(alph*)]
%============= Title Config =============
%\title{\CourseDef\space\TitleDef}
%\author{\AuthorDef \\ \NumberDef}
\begin{document}
\thispagestyle{plain}
{\noindent\Huge\bf  \\[0.5\baselineskip] {\fontfamily{cmr}\selectfont  \TitleDef}         }\\[2\baselineskip] % Title
{ {\bf \fontfamily{cmr}\selectfont \CourseDef}\\ {\textit{\fontfamily{cmr}\selectfont     \today}}}\hspace{150pt}    {\large \textsc{\AuthorDef}} % Author name
\\[1.4\baselineskip]
\vspace{-15pt}

%\begin{figure}[h]%
%    \centering
%    \subfloat{{\includegraphics[scale=0.56]{Kp_Upstream_figure.png} }}%
%    \qquad
%    \subfloat{{\includegraphics[scale=0.56]{Kp_Downstream_figure.png} }}%
%    \caption{Time-averaged $K_{\rho}^U$ (a) and $K_{\rho}^D$ (b) for each simulation. The time average was taken        on the last $54t_0$ ($\sim20$ minutes of simulation time) of the simulation.}%
%    \label{fig:diffus}%
%\end{figure}
%============= Start File =============

\center{\textbf{Abstract}}\\
This experiment attempts and fails to show the violation of Bell's inequality (p.3) in 
the polarizations of down converted photon pairs (p.1-2). The experiment makes use of a 
laser to produce photons, which when incident on a BBO crystal (p. 4) produce photon 
pairs with the same polarization in a process called spontaneous down conversion.

The coincidence window $\tau$ was determined using a linear fit between the product 
of the two detector's single photon counts and the number of raw coincidences (p. 6). 
Due to faulty equipment (p. 8, 12) as well as experimenting with replacing BNC cables, 
APD detectors, and fibre optic cables to try and maximize the amount of detected light (p. 20, 25-28)
resulted in throwing away several data sets (p. 7, 9, 13, 21) to ensure the value of $\tau$ 
was still accurate for my equipment. The final analysis (p. 30) resulted in $\tau =35 \pm 4$ ns
and $\chi^2_{red}=1.3$.\\

After aligning the collectors (p. 10, 30), the goal was preparing an incident beam with polarization $\theta=45^\circ$ from vertical 
and relative phase $\phi=0^{\circ}$ (p. 14-19).
To measure these values both the starred equations on p. 2 and a least squares fit with the 
equation on p. 14 were used. This ultimately resulted in $\theta=47^{\circ}\pm 3^{\circ}$ (p. 31), but any attempt to measure 
$\phi$ was a failure. This is likely due to the massive fluctuations in the corrected coincidences (p. 32).\\

The result was a value of $S=1.1\pm 0.9$ (p. 32), whose calculation is detailed on p. 34. Despite satisfying Bell's inequality, 
this is not evidence for a local hidden variable theory. While we minimized ambient light in the room,
verified the coincidence window (p. 24), and spent several hours troubleshooting equipment to ensure accuracy (p. 24-29), 
we did not explore the effect reorienting the crystal had on the raw coincidence counts.
Thus, there remains a possibility that the proportionally large discrepency between raw and corrected coincidences (p. 32) was 
caused by the failure of incident photons to undergo spontaneous down conversion due to the misorientation of the crystal.
% This would explain both the very small number of raw coincidences, and the large proportion 
% of photons that were removed in the statistical correction of the coincidence window.
\end{document}
