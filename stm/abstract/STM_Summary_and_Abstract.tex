\documentclass[11pt]{article}
%============= Imports =============
\usepackage{physics}
\usepackage{amsmath,amsthm,amssymb}
\usepackage[dvips,letterpaper,margin=1.1in,bottom=0.7in]{geometry}
\usepackage{hyperref}
\usepackage{enumitem}
\usepackage{charter}
\usepackage{tabstackengine}
\usepackage{fancyhdr}
\usepackage{graphicx}
\usepackage{subfig}
\usepackage{siunitx}
\usepackage{float}
\usepackage{listings}
\usepackage{xcolor}
%============= Page style =============
\pagestyle{fancy}
\newcommand{\CourseDef}{PHY424} %Course
\newcommand{\TitleDef}{Observing properties of graphite and gold atomic lattices using the STM} %Title
\newcommand{\NumberDef}{1006154625}
\newcommand{\AuthorDef}{Lucas Prates} %Author (Name)
\rhead{\AuthorDef}
\chead{\TitleDef}
\lhead{\CourseDef}
\definecolor{codegreen}{rgb}{0,0.6,0}
\definecolor{codegray}{rgb}{0.5,0.5,0.5}
\definecolor{codepurple}{rgb}{0.58,0,0.82}
\definecolor{backcolour}{rgb}{0.95,0.95,0.92}
\lstdefinestyle{mystyle}{
    backgroundcolor=\color{backcolour},   
    commentstyle=\color{codegreen},
    keywordstyle=\color{magenta},
    numberstyle=\tiny\color{codegray},
    stringstyle=\color{codepurple},
    basicstyle=\ttfamily\footnotesize,
    breakatwhitespace=false,         
    breaklines=true,                 
    captionpos=b,                    
    keepspaces=true,                 
    numbers=left,                    
    numbersep=5pt,                  
    showspaces=false,                
    showstringspaces=false,
    showtabs=false,                  
    tabsize=2
}

\lstset{style=mystyle}
%============= Matrix Commands =============
\setstackEOL{;}% row separator
\setstackTAB{,}% column separator
\setstacktabbedgap{2ex}% inter-column gap
\setstackgap{L}{1.0\normalbaselineskip}% inter-row baselineskip
\let\bmatrix\bracketMatrixstack
\let\pmatrix\parenMatrixstack
\let\dmatrix\vertMatrixstack
\newcommand{\declarecommand}[1]{\providecommand{#1}{}\renewcommand{#1}}
%============= Other Commands =============
\declarecommand{\R}{\mathbb{R}}
\declarecommand{\Q}{\mathbb{Q}}
\declarecommand{\Z}{\mathbb{Z}}
\declarecommand{\N}{\mathbb{N}}
\declarecommand{\C}{\mathbb{C}}
\declarecommand{\emptyset}{\varnothing}
\newcommand{\mat}[1]{\begin{bmatrix}#1\end{bmatrix}}
%%[label=(alph*)]
%============= Title Config =============
%\title{\CourseDef\space\TitleDef}
%\author{\AuthorDef \\ \NumberDef}
\begin{document}
\thispagestyle{plain}
{\noindent\Huge\bf  \\[0.5\baselineskip] {\fontfamily{cmr}\selectfont  \TitleDef}         }\\[2\baselineskip] % Title
{ {\bf \fontfamily{cmr}\selectfont \CourseDef}\\ {\textit{\fontfamily{cmr}\selectfont     \today}}}\hspace{150pt}    {\large \textsc{\AuthorDef}} % Author name
\\[1.4\baselineskip]
\vspace{-15pt}

%\begin{figure}[h]%
%    \centering
%    \subfloat{{\includegraphics[scale=0.56]{Kp_Upstream_figure.png} }}%
%    \qquad
%    \subfloat{{\includegraphics[scale=0.56]{Kp_Downstream_figure.png} }}%
%    \caption{Time-averaged $K_{\rho}^U$ (a) and $K_{\rho}^D$ (b) for each simulation. The time average was taken        on the last $54t_0$ ($\sim20$ minutes of simulation time) of the simulation.}%
%    \label{fig:diffus}%
%\end{figure}
%============= Start File =============

\textbf{Abstract}\\
This experiment ventured to observe the properties of the graphite and gold atomic 
lattices using the Naio Scanning Tunnelling Microscope. Most of my time with the
equipment was spent determining the ideal parameters to image graphite (pg. 23-32). I did this by 
starting at the 200-500nm scale, zooming in on a recognizable surface feature, and 
changing the parameters to improve the image quality. I repeated this process until
achieving atomic resolution, about a 5nm length scale (pg. 29, image on pg. 79). Measuring the diameter of a single
carbon atom on Image00077 resulted in a value of 113.3 pm (pg. 36-37), although the
uncertainty in this value should be large due to the quality of Image00077 making it 
unclear where the boundariers of each atom are. Further analysis notwithstanding, this 
study can conclude only that the carbon atom has a diameter at the scale of 100pm, which 
seems to agree with the nominal value of a 70pm radius. 

I also spent some time imaging a gold lattice for this experiment (pg. 33-34). I was able 
to find the atomic lattice almost instantly using a scanning time of 0.3s/line, but the
image was extremely distorted.
\end{document}
